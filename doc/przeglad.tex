\chapter{Przegląd istniejących rozwiązań}
\par
Jako że problem zmiany identyfikatorów plików znany już jest od lat, powstało wiele programów próbujących się z~nim uporać. Wiele z~istniejących rozwiązań zostało zaprojektowanych dla plików konkretnego typu lub są modułami większych aplikacji lecz istnieje kilka\footnote{Aplikacje zostały wybrane ze względu na ich popularność i~podejście do rozwiązania problemu} implementacji przeznaczonych do ogólnego zastosowania.
\par
Poniżej zostały przedstawione trzy wybrane implementacje wraz z~subiektywną opinią na ich temat.

\section{Bulk Rename Utility}
\begin{figure}[h]
\begin{center}
\includegraphics[scale=0.75]{img/bulkrename_window.png}
\end{center}
\caption{Okno główne programu Bulk Rename Utility}
\end{figure}

\par
Jednym z~bardziej zaawansowanych i~polecanych programów na platformę Microsoft Windows jest \textit{Bulk Rename Utility}. Aplikacja umożliwia ekstrakcje metadanych z~plików audio (zawartych w~tagach ID3v1) i~obrazów zawierających dane EXIF. Posiada ona także wiele funkcjonalności związanych z~modyfikacją istniejącej nazwy --- takich jak zastępowanie z~użyciem wyrażeń regularnych.
Program wyróżnia się wsparciem dla modyfikacji nazw i~atrybutów katalogów, a~także zwartym interfejsem.
Narzędzie to nie wspiera jednak zmiany kolejności wykonywania działań na nazwie --- wszystkie operacje posiadają stałą pozycje w~kolejce wywołania i~istnieje jedynie możliwość ich włączenia lub wyłączenia.\\

\textit{Bulk~Rename~Utility} posiada również odpowiednik bez interfejsu graficznego --- \textit{Bulk~Rename~Command} --- który jest oddzielnym programem udostępniającym funkcjonalność programu z~linii poleceń (co może znaleźć zastosowanie w~skryptach powłoki).

\section{Métamorphose}
\begin{figure}[h]
\begin{center}
\includegraphics[scale=0.75]{img/metamorphose_window.png}
\end{center}
\caption{Jedna z~zakładek programu Métamorphose}
\end{figure}

\par
\textit{Métamorphose} podobnie jak \textit{Bulk Rename Utility} korzysta z~wbudowanego zestawu funkcjonalności jednak posiada pewne wsparcie dla szablonów nazw plików. Jest również aplikacją wieloplatformową, a~także dzięki zastosowaniu zakładek --- bardziej przejrzystą.

\section{KRename}
\begin{figure}[h]
\begin{center}
\includegraphics[scale=0.55]{img/krename_window.png}
\end{center}
\caption{Konfiguracja szablonu nazwy pliku w~KRename}
\end{figure}

\par
\textit{KRename} w~odróżnieniu od poprzednich programów nie posiada wersji dla systemów Windows. Zestaw zakładek pozwala na znalezienie plików, wybranie akcji do wykonania, a~także przegląd i~edycję wtyczek umożliwiających ekstrakcje danych.\\
Poza trybem edycji szablonu dla nazw plików istnieje prostszy interfejs pozwalający na podstawowe operacje dodania sufiksu lub prefiksu, a~także zmianę wielkości znaków w~nazwie.\\
Zaletą programu jest duży wybór wtyczek pozwalających także na modyfikacje metadanych a~nawet zawarcie w~nowej nazwie rezultatu wywołania kodu JavaScript.

\section{Inne rozwiązania}
\par
Osoby korzystające z~systemów \texttt{POSIX}-owych posiadają ciekawą alternatywę dla jakichkolwiek specjalizowanych aplikacji --- tekstową powłokę zwaną shellem.\\
Nowoczesne powłoki shell takie jak \texttt{zsh} czy \texttt{bash} posiadają funkcjonalności umożliwiające łączenie wyników wywołań wielu komend co wraz z~bogatą liczbą programów dostępnych dla wspomnianych systemów, umożliwia stworzenie polecenia które mogłoby w~prosty sposób modyfikować identyfikatory plików.\\


\begin{lstlisting}[label=shell-rename, caption={Polecenie powłoki zmieniające roszerzenia plików JPEG}]
find ./ -name "*.JPG" -exec rename -v 's/\.JPG/\.jpg/' {} \;
\end{lstlisting}

\par
Na listingu \ref{shell-rename} zostało pokazane przykładowe polecenie powłoki zamieniające rozszerzenia plików JPEG z~'\textit{JPG}' na '\textit{jpg}'.
Wykorzystuje ono dwa programy:
\begin{itemize}
\item \texttt{find} --- znajduje pliki o~rozszerzeniu kończącym się na '\textit{.JPG}' (przez zastosowanie parametru \texttt{-name "*.JPG"})
\item \texttt{rename} --- dokonuje faktycznej zmiany nazwy dla pojedynczego pliku
\end{itemize}
Polecenia realizujące proste zmiany nazw mogą zostać napisane przez średnio-zaawansowanego użytkownika powłoki jednak aby otrzymać bardziej złożone modyfikacje wymagane jest użycie wielu programów co nie jest tak trywialne jak wyżej wymieniony przykład.
