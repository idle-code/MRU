\chapter{Środowisko pracy}
%\label{srodowisko}
%\par
%Rozdział ten zawiera opis środowiska które zostało użyte do stworzenia i testowania implementacji.

\section{Język C++}
\par
Aplikacja \texttt{MRU} została napisana przy użyciu języka C++ w~standardzie z~roku 2003 (ISO/IEC 14882:2003).
Język C++ jest dojrzałym, wieloplatformowym językiem programowania średniego poziomu, używanym od wielu lat przez programistów na całym świecie do tworzenia programów użytkowych, gier, sterowników czy nawet systemów operacyjnych. Dzięki kompatybilności z~C\footnote{C++ nie jest całkowicie kompatybilny z~C, jednak różnice w~obu tych językach są na tyle małe że rzadko wpływają negatywnie na kompatybilność (szczególnie na poziomie ABI).} pozwala na wykorzystanie wielu istniejących bibliotek napisanych zarówno w~C jak i~C++\cite{thinking-in-cpp}.

\subsection{LLVM Clang}
%\begin{wrapfigure}{r}{0.4\textwidth}
%\begin{center}
%\includegraphics[scale=0.70]{img/clang_logo.png}
%\end{center}
%\caption{Logo frontendu Clang}
%\end{wrapfigure}
\par
\texttt{LLVM} --- Low Level Virtual Machine --- jest modułową architekturą do budowy kompilatorów. Pozwala ona na oddzielenie parserów różnych języków programowania od modułu optymalizacji (wspólnych dla wszystkich języków kompilowalnych) i~emiterów kodu bajtowego dla różnych platform.

\par
Clang jest parserem\footnote{Clang jest określany jako '\textit{frontend}' lecz słowo to nie ma dobrego odpowiednika w~języku polskim, a~główną funkcjonalnością tego narzędzia jest właśnie parsowanie plików źródłowych z~kodem C lub C++ do kodu pośredniego LLVM} języków C i~C++ dla architektury \texttt{LLVM}. Projekt jest otwarty (wydawany na licencji BSD) i~zdobywa coraz większą popularność\footnote{Od listopada 2012 Clang wraz z~LLVM stał się domyślnym kompilatorem dla systemu FreeBSD} dorównując, a~nawet przewyższając w~niektórych testach GCC\footnote{GNU Compiler Collection}.

\section{System operacyjny FreeBSD}
%\begin{wrapfigure}{l}{0.3\textwidth}
%\begin{center}
%\includegraphics[scale=0.50]{img/freebsd_logo.png}
%\end{center}
%\caption{Logo systemu FreeBSD}
%\end{wrapfigure}
\par
System FreeBSD jest darmowym i~otwartym systemem operacyjnym z~rodziny BSD wywodzącej się z~rodziny \texttt{UNIX}-ów. Podobnie do dystrybucji GNU/Linux, sam w~sobie wraz z~wieloma, otwartymi bibliotekami tworzonymi przez społeczność stanowi środowisko przyjazne programistom.

\section{Mercurial}
\begin{wrapfigure}{r}{0.3\textwidth}
\begin{center}
\includegraphics[scale=0.50]{img/mercurial_logo.png}
\end{center}
\caption{Logo systemu Mercurial}
\end{wrapfigure}
\par
Do zarządzania plikami źródłowymi oraz kopią zapasową został wykorzystany rozproszony system kontroli wersji Mercurial wraz z~serwisem \url{bitbucket.org}. Narzędzie to pozwala na synchronizację kodów źródłowych między wieloma maszynami, ułatwiając tym samym pracę nad pojedynczym projektem wielu programistów.
\par
W~odróżnieniu od scentralizowanych systemów kontroli wersji takich jak SVN, Mercurial nie wymaga pojedynczego serwera, ani serwera w~ogóle. Pełne repozytorium może być trzymane na każdej maszynie z~której korzysta programista, a~praca różnych programistów może być synchronizowana bezpośrednio między nimi samymi\cite{version-control-example}.

\section{CMake}
\begin{wrapfigure}{r}{0.4\textwidth}
\begin{center}
\includegraphics[scale=0.75]{img/cmake_logo.png}
\end{center}
\caption{Logo narzędzia CMake}
\end{wrapfigure}

\par
Aby projekt był jak najbardziej przenośny i~niezależny od platformy, ważne jest aby jego proces budowania również taki był.
W~celu zapewnienia łatwego wsparcia dla budowania projektu na wielu platformach i~wielu łańcuchach narzędziowych, do budowania MRU został zastosowany CMake --- narzędzie do zarządzania procesem kompilacji i~zależnościami.
\par
CMake pozwala programiście określić z~jakich elementów składa się program i~jakich zewnętrznych zasobów (bibliotek) wymaga. Narzędzie następnie interpretuje skryptowy plik konfiguracyjny i~tworzy natywne dla danej platformy pliki projektowe zawierające odpowiednią do zbudowania projektu konfigurację.

\section{Vim}
\begin{wrapfigure}{r}{0.4\textwidth}
\begin{center}
\includegraphics[scale=0.25]{img/vim_logo.png}
\end{center}
\caption{Logo edytora Vim}
\end{wrapfigure}

\par
Edytor \texttt{Vim} jest rozszerzoną wersją klasycznego edytora \texttt{vi}, który jest standardowym oprogramowaniem w~przypadku dystrybucji zarówno GNU/Linux jak i~systemów z~rodziny BSD. Vim jest też platformą dla wielu pluginów które tworzą jego faktyczną funkcjonalność. Edytor sam w~sobie wspiera pracę z~wieloma dokumentami, koloruję składnie plików źródłowych i~posiada wiele komend ułatwiających produkcję kodu. Dzięki wtyczkom istnieje możliwość rozszerzenia go o~zaawansowane kompletowanie składni czy także szybkie wstawki kodu (ang. snippety).

\clearpage
