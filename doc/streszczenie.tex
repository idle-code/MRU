\clearpage

\begin{center}
\textbf{Streszczenie}
\end{center}

\par
Niniejsza praca inżynierska na temat ''Projekt i implementacja narzędzia do masowej refaktoryzacji identyfikatorów plików z uwzględnieniem ich zawartości'' opisuje projekt architektury oraz implementacje programu Multifile Renaming Utility.
\par
Multifile Renaming Utility --- w skrócie MRU --- jest programem narzędziowym mającym na celu umożliwienie automatycznej zmiany nazw wielu plikom ze względu na metadane w nich zawarte.

\par
Praca jest podzielona na sześć rozdziałów zawierających teoretyczny jak i praktyczny opis problemu i jego rozwiązania.\\
Pierwszy rozdział zawiera wstępny opis projektu i motywacje do jego utworzenia.\\
Drugi rozdział zawiera teoretyczne podstawy problemu oraz opisuje terminologię użytą w pracy.\\
Trzeci rozdział przedstawia przykłady istniejących aplikacji wraz z ich subiektywną oceną pod względem skuteczności w stosunku do przedstawionego problemu.\\
Czwarty rozdział opisuje środowisko pracy, które zostało wykorzystane do stworzenia implementacji.\\
Piąty rozdział zawiera opis architektury, a także szczegóły implementacji gotowej aplikacji.\\
Ostatni, szósty rozdział składa się z wniosków na temat wykonanego projektu.

\vspace*{\baselineskip}

\noindent\textbf{Słowa kluczowe:} system plików, wtyczki, metadane, wxWidgets, boost, SigC++, C++
