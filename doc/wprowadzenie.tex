\chapter{Wstęp}
\label{wstep}

\par
 Z~każdym rokiem ludzie oraz komputery generują coraz większą ilość informacji.
Mimo że duża część z~nich jest przechowywana w~dobrze strukturyzowanych bazach danych, to ciągle, większość ludzi ma bezpośredni dostęp jedynie to tego co przechowuje w~systemie plików własnego komputera.
\par
Od dziesięcioleci dyski twarde pozostają głównym kontenerem danych dla komputerów na całym świecie.
Wiele użytkowników komputera na przestrzeni lat tworzy swoistego rodzaju kolekcje danych --- albumy zdjęć, biblioteki muzyczne czy filmowe, a~także duże ilości dokumentów na potrzeby działalności gospodarczej czy też prywatnej, które stopniowo zastępują starsze formy gromadzenia informacji. Dodatkowo, niektórzy administratorzy zarządzający serwerami aplikacji zmagają się z~problemem wielkiej ilości plików generowanych przez użytkowników ich systemów.

\section{Motywacja}
\label{motywacja}
\par
Tysiące plików mogą stworzyć gąszcz informacyjny w~którym człowiek będzie czuł się zagubiony. Przy coraz większej ilości informacji nie bez znaczenia pozostaje czynnik ludzki którego możliwości percepcji są ograniczone.
Istnieje wiele programów ułatwiających katalogowanie danych jednak skierowane są one często na pojedyncze typy plików, a~także wymagają od użytkownika przyzwyczajenia się do ich używania. Z~drugiej strony, przeciętny użytkownik jest zwykle przyzwyczajony do standardowego programu oferowanego przez większość systemów operacyjnych --- przeglądarki plików.\\

Problemem jednak jest fakt że programy rzadko generują przyjazne użytkownikowi nazwy plików, zwykle ograniczając się do prostego prefiksu i~grupy kolejnych numerów zapewniających unikalność.
Doświadczyły tego na pewno osoby przenoszące dane z~kamer cyfrowych, w~których po każdym usunięciu danych z~pamięci urządzenia, kolejne zdjęcia numerowane są od początku. Może doprowadzić to do konfliktu przy kopiowaniu wielu sesji do tego samego katalogu na dysku lub nawet nadpisaniu starszych obrazów.

Najprostszym rozwiązaniem z~punktu widzenia programisty możne być zrzucenie obowiązku wyboru nazwy na samego użytkownika, jednak rzadko jest to dobrym rozwiązaniem gdyż w~dzisiejszym świecie liczy się szybkość działania, a~zmuszanie użytkownika do myślenia nigdy nie przyspiesza jego działań.

%Rzadko można również znaleźć interesujący utwór w~bibliotece muzycznej której pliki posiadają nazwy różniące się jedynie numerem ścieżki.\\
%Wreszcie istnieją też sytuacje gdy wiele różnych plików jest trzymanych w~pojedynczym katalogu co skutecznie utrudnia nawigację i~znalezienie tego czego użytkownik faktycznie poszukuje.

\section{Cel i~zakres pracy}
\par
Celem niniejszej pracy inżynierskiej jest stworzenie programu narzędziowego pozwalającego na automatyczne generowanie identyfikatorów plików na podstawie metadanych w~nich zawartych, a~także ich stosowanie do zbiorów plików wybranych przez użytkownika.

\par
Zakres pracy obejmuje:
\begin{itemize}
\item Przegląd istniejących rozwiązań - programów i~technik wspomagających masową zmianę identyfikatorów plików.
\item Porównanie funkcjonalności istniejących narzędzi i~ich ograniczeń.
\item Projekt oraz implementacja wieloplatformowej architektury modułów.
\item Stworzenie parsera wyrażeń zawierających metatagi.
\item Projekt graficznego interfejsu użytkownika opartego na bibliotece wxWidgets.
%\item Implementacja backendu do systemu plików opartego na bibliotece boost::filesystem.
\item Implementacja przykładowych modułów metatagów.
\item Testy aplikacji.
\end{itemize}

\section{Założenia}
\label{zalozenia}
Gotowa aplikacja powinna być niezależna od systemu operacyjnego w~stopniu w~jakim pozwalają na to zależności użytych bibliotek. Dzięki modułowej budowie powinna także udostępniać interfejs pozwalający na jej łatwą rozbudowę bez ingerencji w~istniejący kod źródłowy.

\section{Plan pracy}
\label{plan-pracy}
W~rozdziale \ref{teoria} opisano teoretyczny schemat przechowywania danych oraz genezę systemów oraz identyfikacji plików. Rozdział \ref{srodowisko} zawiera opis środowiska wykorzystanego do stworzenia projektu i~implementacji aplikacji.
Rozdział \ref{implementacja} zawiera szczegółowy opis architektury aplikacji wraz z~rozwiązaniami wykorzystanymi do jej stworzenia.
Ostatni rozdział --- \ref{wnioski} --- zawiera wnioski na temat wykonanego projektu.

\clearpage
