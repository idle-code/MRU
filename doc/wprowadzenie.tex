\chapter{Wstęp}
\label{wstep}

\par
Z każdym rokiem ludzie oraz same komputery generują coraz większą ilość informacji.
Mimo że duża część z nich jest przechowywana w dobrze strukturyzowanych bazach danych, to ciągle, większość ludzi ma bezpośredni dostęp jedynie to tego co przechowuje w systemie plików własnego komputera.
\par
Od dziesięcioleci dysk twardy pozostaje głównym kontenerem danych dla komputerów na całym świecie.
Wiele osób na przestrzeni lat tworzy swoistego rodzaju kolekcje danych --- albumy zdjęć, biblioteki muzyczne czy filmowe, a także duże ilości dokumentów na potrzeby działalności gospodarczej czy prywatnej pracy. Niektórzy administratorzy zarządzający serwerami aplikacji zmagają się z problemem wielkiej ilości plików generowanych przez użytkowników.

\section{Motywacja}
\label{motywacja}
\par
Tysiące plików mogą stworzyć gąszcz informacyjny w którym człowiek będzie czuł się zagubiony. Przy coraz większej ilości informacji nie bez znaczenia pozostaje czynnik ludzki którego możliwości percepcji są ograniczone.
Istnieje wiele programów ułatwiających katalogowanie danych jednak skierowane są one zwykle na pojedyncze typy plików i wymagają od użytkownika przyzwyczajenia się do ich używania. Z drugiej strony, przeciętny użytkownik jest zwykle przyzwyczajony do standardowego programu systemów operacyjnych --- przeglądarki plików.\\

Problemem jednak jest fakt że programy rzadko generują przyjazne użytkownikowi nazwy plików, zwykle ograniczając się do prostego prefiksu i grupy kolejnych numerów zapewniających unikalność.\\
W przypadku obrazów, często bywa iż kolekcja musi być trzymana w wielu katalogach ponieważ nazwy plików się pokrywają.\\
Rzadko można również znaleźć interesujący utwór w bibliotece muzycznej której pliki posiadają nazwy różniące się jedynie numerem ścieżki.\\
Wreszcie istnieją też sytuacje gdy wiele różnych plików jest trzymanych w pojedynczym katalogu co skutecznie utrudnia nawigację i znalezienie tego czego użytkownik faktycznie poszukuje.

\section{Cel i zakres pracy}
\par
Niniejsza praca ma na celu stworzenie programu narzędziowego pozwalającego na automatyczne generowanie identyfikatorów (nazw) plików na podstawie (meta)danych w nich zawartych, a także ich stosowanie do zbiorów plików.

\par
Zakres pracy obejmuje:
\begin{itemize}
\item Przegląd istniejących rozwiązań - programów i technik wspomagających masową zmianę identyfikatorów plików.
\item Porównanie funkcjonalności istniejących narzędzi i ich ograniczeń.
\item Projekt oraz implementacja wieloplatformowej architektury modułów.
\item Stworzenie parsera wyrażeń zawierających metatagi.
\item Projekt graficznego interfejsu użytkownika opartego na bibliotece wxWidgets.
\item Implementacja backendu do systemu plików opartego na bibliotece boost::filesystem.
\item Implementacja przykładowych modułów metatagów.
\item Testy aplikacji.
\end{itemize}

\section{Założenia}
\label{zalozenia}
Gotowa aplikacja powinna być niezależna od systemu operacyjnego w stopniu w jakim pozwalają na to zależności użytych bibliotek. Dzięki modułowej budowie powinna także udostępniać interfejs pozwalający na jej łatwą rozbudowę.

\section{Plan pracy}
\label{plan-pracy}
W rozdziale \ref{teoria} opisano teoretyczny schemat przechowywania danych oraz genezę systemów oraz identyfikacji plików. Rozdział \ref{srodowisko} zawiera opis środowiska wykorzystanego do stworzenia projektu i implementacji aplikacji.
Rozdział \ref{implementacja} zawiera szczegółowy opis architektury aplikacji wraz z rozwiązaniami wykorzystanymi do jej stworzenia.
Ostatni rozdział --- \ref{wnioski} --- zawiera wnioski na temat wykonanego projektu.

\clearpage
