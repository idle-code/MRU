\chapter{Wstęp}

\par
Z każdym rokiem ludzie oraz same komputery generują coraz większą ilość informacji.
Mimo że duża część z nich jest przechowywana w dobrze strukturyzowanych bazach danych, to ciągle, większość ludzi ma bezpośredni dostęp jedynie to tego co przechowuje w systemie plików własnego komputera.
<Rodzaje danych, ich zastosowanie>

\par
Od dziesięcioleci dysk twardy pozostaje głównym kontenerem dla danych użytkowników komputerów na całym świecie. 
<Dane przechowywane w systemach plików>

\par
<Systemy plików, ich cechy wspólne, ograniczenia, metadane>

\section{Motywacja}
<problem z identyfikatorami>

\section{Cel i zakres pracy}
\par
Niniejsza praca ma na celu stworzenie programu narzędziowego pozwalającego na automatyczne generowanie identyfikatorów (nazw) plików na podstawie (meta)danych w nich zawartych, a także ich stosowanie do zbiorów plików.

\par
Zakres pracy obejmuje:
\begin{itemize}
\item Przegląd istniejących rozwiązań - programów i technik wspomagających masową zmianę identyfikatorów plików.
\item Porównanie funkcjonalności istniejących narzędzi i ich ograniczeń.
\item Projekt oraz implementacja wieloplatformowej architektury modułów.
\item Stworzenie parsera wyrażeń zawierających metatagi.
\item Projekt graficznego interfejsu użytkownika opartego na bibliotece wxWidgets.
\item Implementacja backendu do systemu plików opartego na bibliotece boost::filesystem.
\item Implementacja przykładowych modułów metatagów.
\item Testy aplikacji.
\end{itemize}

\section{Założenia}
Gotowa aplikacja powinna być niezależna od systemu operacyjnego w stopniu w jakim pozwalają na to zależności użytych bibliotek. Dzięki modułowej budowie powinna także udostępniać interfejs pozwalający na jej łatwą rozbudowę.

\section{Plan pracy}
W rozdziale \ref{teoria} opisano teoretyczny schemat przechowywania danych oraz genezę systemów oraz identyfikacji plików. Rozdział \ref{srodowisko} zawiera opis środowiska wykorzystanego do stworzenia projektu i implementacji aplikacji.
Rozdział \ref{implementacja} zawiera szczegółowy opis architektury aplikacji wraz z rozwiązaniami wykorzystanymi do jej stworzenia.
Ostatni rozdział --- \ref{wnioski} --- zawiera wnioski na temat wykonanego projektu.

\clearpage
