\chapter{Wnioski}
\label{wnioski}

\par
Wiele bibliotek i~szczegółów implementacji sprawiło że projekt pracy inżynierskiej okazał się nieco trudniejszy niż było to przewidywane.
Dużą częścią pracy stanowiło połączenie istniejących bibliotek i~technologii aby mogły ze sobą współpracować. Różnica zaawansowania oraz stylów interfejsów użytych narzędzi wymusiła tworzenie dodatkowych abstrakcji lecz dzięki temu, pozwoliła także na zmniejszenie zależności między-modułowych co zaowocowało powstaniem aplikacji o~dużych możliwościach rozwoju.
%Omówiona aplikacja może być dalej rozwijana jako projekt open-source.

\par
Również projekt architektury aplikacji okazał się nie tak prosty jak mogłoby to wynikać z założeń. Aby móc stosować operacje działające na wielkich zbiorach danych, przepisany musiał zostać cały system wtyczek wejścia, który w pierwszej po prostu wczytywał listę plików, a w ostatecznej był kaskadą iteratorów.

\par
Nad wyraz ciekawym doświadczeniem okazał się również parser metawyrażeń.
W pierwszej wersji został on zaimplementowany w sposób spójny i monolityczny, jednak gdy zaszła potrzeba jego rozszerzenia, musiał zostać podzielony na bardziej niezależne moduły (tokenizera, leksera i samego parsera). Zaowocowało to ciekawą architekturą, którą autor planuje rozwijać w przyszłości.

\par
Stworzona aplikacja sprostała założeniom pod które została zaprojektowana i~stanowi dzięki temu użyteczne narzędzie które może pozwolić ludziom na to na co zostały stworzone komputery --- zautomatyzowanie monotonnych czynności i~przyspieszenie pracy.

\par
Dodatkową zaletą wykonanej aplikacji jest fakt że niektóre jej części (takie jak menadżer wtyczek) dzięki swojej uniwersalności mogą posłużyć do budowy kolejnych programów.
